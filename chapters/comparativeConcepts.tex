\Chapter{Comparative concept and counterpart}{Comparing languages}
\label{ch:languagecomparison}

To be able to compare data from different languages, it is necessary to make sure that the data is comparable across languages. It is crucial to realise that grammatical terminology like `subject', `genitive' or `diminutive' are not sufficiently defined by tradition to be automatically useful across languages.\footnote{The most forceful recent plea against the assumption that such terms are automatically well-defined is Haspelmath (2010). However, the necessity to explicitly define the topic of a cross-linguistic investigation has been recognised throughout recent decades of scholarship. REFERENCES.} The central prerequisite for comparability is an explicit definition of the topic to be compared.

\concept{domain}{
A \textsc{\textbf{domain}} is a cross-linguistic topic of investigation. A domain consists of three different parts, namely (i) a name for the domain, (ii) an explicit definition, and (iii) the actual data selected for comparison. The name is called \textsc{domain label}, the explicit definition is called \textsc{comparative concept}\refc{comparativeconcept}, and the data selected for comparison is called \textsc{counterpart}\refc{counterpart}.
}
%
Names for domains are just labels that have to be explicitly defined anew for every comparative investigation. Unfortunately, the same labels are often used both for language-specific constructions\refc{construction} as well as for cross-linguistic domains\refc{domain}. To avoid confusion between the two, it is good practice to clearly distinguish \textsc{domain labels} from \textsc{construction labels}. When writing in English, it is possible to use capitalisation to differentiate the two. Cross-linguistic domain labels are commonly written in lowercase (e.g. `diminutive'), while language-specific construction labels start with a capital letter (e.g. `Diminutive'). To prevent any possible confusion, it is strongly preferred to additionally add a language name\refc{glossonym} to a construction label (e.g. `Dutch Diminutive').\footnote{This tradition originated with Comrie (1976). This capitalisation trick cannot be used in all orthographic traditions. For example, German nouns always have to be capitalised, so it is not possible to use a lowercase nouns in regular German orthography. The addition of a language name becomes crucial in such situations.}

\concept{subdomain}{
Two domains are \textsc{\textbf{subdomains}} of an overarching domain when languages recurrently use the same counterpart\refc{counterpart} to express both subdomains.
}
%
Cross-linguistic research often investigates the relation between various domains. Or, formulated differently, it investigates the structure of subdomains within a single domain\refc{pattern}.  For example, the domains `instrumental' (how does a language express that something is used as an instrument) and `accompaniment' (how does a language express that a participant is accompanying another participant) are strongly linked into an overarching domain, because many languages use the same construction for both (e.g. English \textit{with}).\footnote{Subdomains are also known in the literature as \textsc{senses} (Haspelmath 2003), \textsc{etic grid} (Levinson \& Meira 2003) or \textsc{analytical primitives} (Cysouw 2010).}

In current linguistics, the \textsc{optimal domain hypothesis} seems to be widespread. This hypothesis says that it is possible to define a finite number of domains that are necessary and sufficient for the analysis of all human languages.\footnote{This hypothesis is approximately the same as \textsc{categorical universalism} from Haspelmath (2010:663).} Whether this hypothesis holds is an open question. There are at least two different ways in which it might not be true. First, there might turn out to be just one domain, i.e. all domains proposed might turn out to be subdomains of one overarching domain. Or, second, there might turn out to be an infinite number of sensible domains.\footnote{Note that the observation that there is more than one sensible definition of the term `tense' (an example discussed in Haspelmath 2010:679) does not contradict this hypothesis. Maybe linguists can agree at some point that we need, say, 17 different definitions of `tense' for a full fledged analysis of human language (though hopefully using different labels), then the hypothesis is still true.}

\concept{comparativeconcept}{
A \textsc{\textbf{comparative concept}} is a definition of a domain\refc{domain}, which is primarily meaning/function-based definition, but which frequently includes additional form-based constraints. This definition is used to select \textsc{constructions}\refc{construction} from the different languages to be compared. The set of constructions selected from a single language is called a \textsc{counterpart}\refc{counterpart}.\footnote{The term \textsc{comparative concept} was proposed by Haspelmath (2010). In much of the typological literature, the term \textsc{tertium comparationis} is used with the same intention.}
}
%
A comparative concept is necessarily universally applicable across all human languages. If it turns out that a specific comparative concept is not applicable in certain languages, then the definition has to be revised. This approach can be called the \textsc{principle of universally applicable domain}. To make them universally applicable, comparative concepts are mostly based on meaning/function. Such definitions are most easily applicable across the widely different constructions as attested in the world's languages. Additional form-based constraints can be used, but care should be taken to formulate such constraints in a universally applicable manner. There appear to be only very few universally applicable form-based criteria, possibly not much more than (i) size of construction, (ii) number of counterparts, (iii) relative position construction parts, and (iv) extent of morphological fusion of parts\refc{shape}. In the practice of worldwide language comparison, the process of formulating and adjusting the definition of a domain is a central part of proper research. The details of the definition itself are often more important than their actual empirical application.

Further, there is a \textsc{principle of unrestricted domain choice}, which simply means that there are no \textit{a-priori} reasons to restrict the kind of definitions that are allowed. It is important to realise that this principle does not imply that every method is equally suitable. However, whether a definition is useful can only be judged \textit{ex post}, i.e. by judging the usefulness of any insights that arise from the definition.\footnote{Haspelmath's (2010) \textsc{categorical particularism} seems to be a similar principle.} There are various \textit{ex post} reasons that might be brought up as an argument for a specific comparative concept. For example, someone might argue that comparative concepts should be chosen in such a way that the counterparts form a coherent category in all individual languages. The claim that this is possible might be called \textsc{homogeneous counterpart assumption}. A different approach might argue that definitions of domains should be based on (non-linguistic) insights into the functioning of the neurological-cognitive phase of language \refc{form}. This might be called the \textsc{cognitive grounding assumption}. 

% a well-defined domain is a universal in some sense conceptual universal. linguistic universal when counterparts form homogenous group (utterance negation?). This principle of course does not imply that a properly defined domain itself is an interesting universal property of human language. 

\concept{counterpart}{
A \textsc{\textbf{counterpart}} is a set of language-specific constructions \refc{construction} --- though in practice often just a single construction --- that is selected from a language on the basis of a comparative concept\refc{comparativeconcept}.
}
%
A comparative concept is a definition that specifies how to select constructions (possibly one, possibly more than one) from a language. The resulting set of constructions for each language will be called a \textsc{counterpart}.\footnote{The term \textsc{counterpart} is proposed by Good.} The central procedure of language comparison is to compare counterparts. The technique to compare counterparts across languages is not automatically obvious, because counterparts are language-specific constructions (for a full discussion, see Chapter~\ref{ch:typology}).

In the field of linguistic typology, the counterparts are often extracted from reference grammars or by using questionnaires to be filled out by specialists in the  languages investigated. The problem to properly define comparative concepts and select counterparts can also be solved by using experimental stimuli with visual context (e.g. pictures or movie clips), or by using linguistically expressed context in the form of parallel texts.

Comparative concepts and counterparts are also used in other subfields of linguistics. In quantitative historical linguistics, the `Swadesh-style' \textsc{wordlist approach} is widespread. Using the terminology as developed in this chapter, the term `wordlist' can now be clarified. The starting point is a set of lexical comparative concepts, or \textsc{conceptlist}. In most practical instances, these comparative concepts are rather ill-defined, often just relying on individual English (or Spanish or Russian) words with minimal explicit definition. This is the same fallacy as discussed at the start of this chapter: assuming that domain labels are sufficiently defined automatically --- which they mostly are not. A \textsc{wordlist} is then the list of counterparts in a specific language.
