\Chapter{Form and meaning}{Setting the perspective on language}

\concept{language}{
The term \textsc{\textbf{language}} (without preceding determiner) refers to the phenomenon studied in the field of \textsc{linguistics}, while \textsc{\textbf{a language}} (with preceding determiner) refers to a specific instantiation of language, which is the (primary) object of \textsc{philology}.
}
%
The conceptual distinction made here between the fields of \textsc{linguistics} and \textsc{philology} is introduced here to clearly separate specific goals of scholarship. Both approaches are equally needed to obtain further insight in activities like speaking, listening, signing, writing, reading, etcetera. Scholars of these topics never seem to be exclusively a linguist or a philologist. Everybody seems to be both at the same time, though the majority appears to put an emphasis on the philological side. This emphasis is deservedly so, because without philology (i.e.~in-depth study and analysis of actual utterances in specific languages) linguistics (i.e.~any higher abstractions on language in general) would be neigh impossible. This meta-methodological essay deals with linguistics, but the explicit assumption is that it is only possible to obtain more general insights into language by studying many different languages and comparing their differences and similarities. 

% Chapter~\ref{ch:language} deals with the perennial problem of defining what actually is `a language' and Chapter~\ref{ch:languagespecific} discusses some general concepts related to the study of individual languages.

\concept{perspective}{
Two perspectives on \textsc{language}\refc{language} can be distinguished. Language can be investigated both from a \textsc{\textbf{neurological-cognitive perspective}} (or `inside-human' perspective), which includes the pervasive human qualia of \textsc{meaning}\refc{meaning},  as well as from a \textsc{\textbf{physical-linguistic perspective}} (or `outside-human' perspective), which includes the many facets of \textsc{linguistic form}\refc{form}.
}
%
Both the neurological-cognitive perspective (i.e.~`what happens inside a human being when language is used') and the physical-linguistic perspective (i.e~`what is the structure of language when it is passed between humans') are crucial to obtain a full understanding of language. These two perspectives can be seen as investigating two different phases of language. Both phases of language exist simultaneously and influence each other, both are necessary for language to function, and none can exist without the other. Yet, the perspective on language in this essay will be strongly biased towards the physical-linguistic perspective, i.e.~the physical \textsc{form} of language.

What happens during the neurological-cognitive phase is not easy to observe. In the fields of psycholinguistics and neurolinguistics a large arsenal of methods has been developed to tackle questions about what happens `inside' a human being when language is used. However, any data about this phase of language remains hard to obtain, requiring extensive machinery and highly sophisticated experimental methodologies. In contrast, the physical-linguistic phase of language is extremely easy to observe. Humans do not seem to want to stop exchanging language, be it in the form of sound waves, written language, sign language or in any other physical instantiation.

\concept{form}{
The \textsc{\textbf{form}} of language is any physical instantiation (outside the human body) of language.
}
%

\concept{meaning}{
The terms \textsc{meaning} (mostly used for lexical elements), \textsc{function} (mostly used for grammatical morphemes and constructions), and \textsc{sense} (mostly used for subdivisions of meanings/functions) are not differentiated here. They are all equated here with \textsc{\textbf{extent of use}}.
}
%
The experience of meaning is shared among all humans: the definitive feeling that we understand and know what is the meaning of some sound waves entering our ears, or some scribbled symbols seen through our eyes. This experience seems so natural and easy, and we seem to agree

This works because language is taken here as an `outside-human' object

Meaning is the big problem: we have just a vague grasp of what meaning really is. Forms and their extent of use are much more empirical concepts to work with

semasiological vs.~onomasiological approach: Because we don't know too much about meaning, mostly semasiological


\concept{homonomy}{
The terms \textsc{polysemy} (i.e.~different but related meanings are expressed by the same form), \textsc{homonomy} (i.e.~one form expresses a range of unrelated different meanings) and \textsc{syncretism} (i.e.~one form expresses multiple meanings which had different forms in an earlier stage of the language) are not differentiated here. They are simply taken to be cases of \textsc{\textbf{one form with multiple uses}}.	
}
%
assumption: every form is homonymous/polysemous!

homonomy has the intuition of `unrelated' meanings being combined into one form. However, it is very difficult to decide on `similarity' between meanings. See semantic maps!

(theory of utterance selection, exemplar approach, etic/emic distinction)

[notetoself] dogma (basic assumption underlying a large body of scholarship)- assumption (defining property of what something means) --- hypothesis (note wikipedia: ``a proposed explanation for a phenomenon'', so I actually mean `prediction' or `expectation', as according to wikipedia ``Any useful hypothesis will enable predictions by reasoning'').

Also: generalizations/parameters --- model --- theory 
