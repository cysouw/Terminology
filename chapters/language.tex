\Chapter{Doculect and languoid}{Disentangling the concept of a language}
\label{ch:language}

Although it completely obvious to everybody that there are numerous different languages spoken in this world, it turns out that it is far from trivial to draw consistent and useful limits between what should counts as a single language and what should count as different languages. Now, it is perfectly reasonable for laypeople and non-linguistic scholars to use names for different languages without reflecting on the proper definition of the objects referred to by these names. Simply using a name like English or Witotoan suffices as an informal communicative designation for a particular language or a language group. However, for the linguistics community, which is by definition occupied with the details of languages and language variation, it is somewhat bizarre that there does not exist a proper technical apparatus to talk about intricate differences in opinion about the precise sense of a name like English or Witotoan when used in academic discussion. 

In this chapter I will discuss three interrelated concepts, \textsc{languoid}\refc{languoid}, \textsc{doculect}\refc{doculect}, and \textsc{glossonym}\refc{glossonym}, which provide a principled basis for discussion of different points of view about key issues, such as whether two varieties should be associated with the same language, and allow for a precise description of what exactly is being claimed by the use of a given genealogical or areal group name.\footnote{This chapter is based on~\citet{cysouwgood2013}. For a more technical discussion and references, please see the extensive discussion there.} However, before turning to these definitions, it is important to realise that there are various possible approaches to the definition of what counts as a language. I will here somewhat artificially oppose two different approaches (ignoring others), based on the question why there exist different languages in the first place. The answer to this question is that language fulfils two opposing functions.

\concept{communication}{
A central function of a language is \textbf{\textsc{communcation}}. In this function a language is considered as an instrument to transfer information between humans.
}
%
The idea that a language is a communicative device is often seen as its prime \textit{raison d'être}. Now, languages are undoubtedly highly efficient instruments to exchange messages, explain actions, and relate events that others did not witness. However, would such communication be the sole function of language, than I see no reason why there is this universal tendency of languages to diversify. The observation that languages always change and diversify is so pervasive that in linguistics the\textsc{universality of language change} can be considered a foundational dogma of the field. However, the communicative function acts as a unifying force on linguistic diversity, leading to less diversity and opposing change. 

\concept{identification}{
A second central function of a language is \textbf{\textsc{identification}}. In this function a language is considered as a socio-political instrument to negotiate ingroup/outgroup status in society.
}
%
When I say that languages are also social instruments, then I am not referring to the widespread use of languages to recount the latest gossip about mutual friends and foes, but to a much deeper and automatic process that goes along with the use of a language. Universally, humans seem to have strong opinion about `right' and `wrong ' language usage, and these opinions are automatically and constantly (and mostly completely unconsciously) used to judge others as being one of your own kind or not. Not only do we use cues on all levels of linguistic structure (phonetics, lexicon, morphosyntax, semantics, pragmatics) to make such judgements, we also convey them in our own speech. In a sense, we constantly show who we are by speaking, just like we convey our identity by the clothes we wear, our haircut, or the food that we prefer to eat in public.






\concept{glossonym}{
\textsc{\textbf{glossonym}}
}

\concept{doculect}{
\textsc{\textbf{doculect}}
}

\concept{languoid}{
\textsc{\textbf{languoid}}
}