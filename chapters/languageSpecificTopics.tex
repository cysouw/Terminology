\Chapter{Utterance and construction}{Language-specific meta-concepts}
\label{ch:languagespecific}

The central focus of most linguistics is the investigation of the structures of individual languages. At the same time, the larger aim of most linguistic research is to rise above the details of single languages and formulate theories about human language in general. Yet, the main tenet of this book is that these two goals should conceptually be clearly separated. This chapter introduces the central language-specific meta-concepts.

\concept{utterance}{
An \textsc{\textbf{utterance}} is an concrete well-formed instantiation of language usage actually attested in a \textsc{doculect} \refc{doculect}, normally being a complex construal of various interconnected \textsc{constructions} \refc{construction}.
}
%
Utterances are the basic data for any linguistic investigation. An utterance can be very short and simple (e.g. just saying \emph{yes} in English), but it normally is a complex hierarchical combination of various interconnected constructions. 

By definition, an utterance is a one-of-a-kind affair, as each utterance is actually uttered in a concrete time and place, and its interpretation can only be assessed in relation to the specific situation of utterance. In practice, the field of linguistics often abstracts away from this detail and assumes that it is possible to investigate an utterance in isolation or with only minimal context.\footnote{There is a growing awareness that linguistics can also profitably be directly based on individual utterances, or \textsc{exemplars}. Such a token-based approach is obviously taken in research using corpora. In the context of language change, see the \textsc{theory of utterance selection} (Croft 2000)} Such an abstract utterance might be called a \textsc{sentence}. In many cases this abstraction does not lead to unwarranted interpretations, but it amounts to a (possibly misleading) simplification. This is most clearly exemplified by the widespread occurrence of sentences with many different interpretations, depending on the context of utterance. Utterances are almost never ambiguous, in contrast to sentences.

Utterances are in principle always `well-formed', i.e. every attested utterance is a valid object of (primary) language to be studied in linguistics (i.e. in secondary language). This assumption might be referred to as the \textsc{descriptional dogma} of modern linguistics. Note that there is a bewildering number of reasons a speaker might bring up to disqualify an utterance as being `not-well-formed.' In principle, each such reason is a respectable (secondary) reflexion on the speaker's own language, which is a piece of information that should always be taken serious by linguistic theory (even if eventually dismissed by explicit argumentation). Also note that in much of modern linguistics the speaker and the linguist are the same person. This combination makes utterance and reflection much closer and quicker, though it also includes the danger of influence from theoretical expectation on the reflection.

Utterances belong to a specific doculect (`language'), i.e. utterances are language-specific. However, note that a doculect might of course be a mixed language, or an instance of code-switching, which can lead to different `languages' being combined into a single utterance. This will still be called `language-specific', actually only to prevent the more cumbersome-sounding `doculect-specific' \refc{doculect}.

\concept{construction}{
A \textsc{\textbf{construction}} is an abstract language-specific combination of \textsc{coding devices} \refc{codingdevice} and other embedded constructions (typically including \textsc{slots} \refc{slot} to be filled by embedded constructions), which is used recurrently with a reasonably coherent set of meanings/functions.
}
%
The term `construction' is used in a bewildering number of meanings in current linguistics, so there is a great potential for confusion when using this word. Still, I have decided to retain this term here, although the current conceptualisation will add yet another variation to the meaning of this word.\footnote{The current concept of constructions is close the the one in Croft (2001). It also seems to be basically same as Haspelmath's (2010:664) \textsc{descriptive categories}.}

A construction will here be understood to be a combination of language-specific elements that are used recurrently in a particular language. So, in its most simple instantiation, a construction can be a word, a morpheme, or a stock phrase. More interestingly, constructions are often combinations of such elements including `slots' to be filled (possibly recursively) by other elements. For example, the English Progressive is a construction consisting of a form of the verb \emph{to be} together with a slot for a verb, which has a suffix \emph{-ing}.

Actually, all language-specific grammar can be seen as such nested constructions, and I think that it is possible to profitably discuss divergent proposals as made in different approached to grammar in current linguistics, and reach consensus in the majority of cases. Probably the most difficult issue preventing consensus in current linguistics is the question how much unification of construction is deemed necessary. Constructions are always abstractions over actual variation, but some (more `descriptional') linguists might prefer separate listing of less abstract constructions, even when they are only slightly different, while others (more `theoretical') linguists would rather unify many constructions into one underlying more abstract construct, often invoking some kind of movement to derive the actual constructions.

There are a few important definitional characteristics of the current conceptualisation of constructions. First, all constructions are language-specific. By current definition there are no constructions that occur in more than one language --- but of course note the difficulty of what actually counts as a single language \refc{languoid}. Not all is lost, though. There are various possible similarities between constructions from different languages: they might be homologous \refc{homology}, or they might belong to a similar strategy \refc{strategy} or pattern \refc{pattern}. But they are never the same construction. Second, constructions are linguistic abstractions, i.e. they are secondary language. Instances of constructions are used in actual utterances, or, formulated reversely, a construction is a grouping of all occurrences in actual utterances. Third, the identification of a construction is not self-evident. I would even coin the proper recognition of a constructions to be a `discovery.' Scholars will and should discuss about the preferred description of any construction; reconsider, reformulate, and refute them until ideally a consensus arises.

Finally, as stated in the definition above, constructions are normally expected to express a `reasonably coherent' set of meanings. The reason for this rather vague formulation is that it is extremely difficult (if not impossible) to propose a general analysis of meaning on the basis of just a single language \refc{meaning}.

\concept{codingdevice}{
A \textsc{\textbf{coding device}} is an abstract minimal functional construction of a specific language.
}
%
The term `coding device' is proposed here for a slightly more general conceptualisation than the traditional notion of a morpheme, i.e. the minimal meaning-bearing (or function-bearing) unit of language. Typically, coding devices are individual morphemes or lexemes, but they can also be non-linear morphology (tonemes, ablaut, umlaut, metathesis, etc.) or even specification of relative ordering of parts, and amount of morphological fusion between parts.

In empirical practice, coding devices are the minimal building blocks necessary to formulate all proposed constructions. Formulated differently, coding devices are the endpoints (`leaves') of the hierarchical structure of constructions in sentences (`trees'). As such, all coding devices are also constructions, and the introduction of this new term is mainly for convenience.

There is a widespread assumption in linguistics that each utterance can be broken down into a disjunct (`non-overlapping') set of coding devices. This assumption that there is a single underlying separation of each utterance into a set of smallest building blocks can be called the \textsc{single-level coding hypothesis}.

\concept{slot}{
A \textsc{\textbf{slot}} is a part of a construction that \emph{defines} a class of constructions (typically lexemes), namely, all constructions that can fill the slot belong to the class.
}
%
The idea of a slot in a construction is often defined reversely. By assuming that a language has classes of coding devices (e.g. `word classes' or `inflectional classes'), then a slot can be characterised by a specific such class of elements that are allowed to occur in the slot. The problem with that approach is that there is no independent definition of the classes themselves. By reversing the definition (as proposed here), classes are \emph{defined} by a specific slot in a construction. That implies that each slot in each construction \emph{a priori} defines a separate class, and it becomes an empirical question how tightly all these classes fit together into a single overarching word class division for the whole language. The assumption that each language has a small set of overarching classes can be called the \textsc{word class hypothesis}.

\concept{behavior}{
The \textsc{\textbf{behavioral potential}} of a construction (typically a lexeme) is defined as the collection of slots that the construction can fill.
}
%
Behavioral potential is 

