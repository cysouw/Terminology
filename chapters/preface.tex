This book is written in \textsc{quaternary language}, that is, these sentences your are reading are language about language about language about language.



\ \newline To start from the beginning, \textsc{primary language} is all language that is considered as object of linguistic study, i.e.~the parts that are typically typeset in italics. Up one level, the language as used in the field of linguistics is language that discusses such primary language, so any linguistic theory, including all grammatical terminology, is \textsc{secondary language} (`language about language'). In contrast, the topic of this book is \textsc{tertiary language}, namely words (and the underlying conceptual ideas) that are important for formulating a linguistic theory. Thus, because the topic of this book is tertiary language, the sentences that I use in this book itself to elucidate these proposals are `language about tertiary language', i.e.~quaternary language.

From this perspective, this book is a contribution to a meta-theory for linguistics. So, I will not be discussing terms like `adjective' or `nasality', i.e.~the actual bread-and-butter of linguistic theory. Instead, I will present a collection of abstract meta-concepts that I consider important to be clarified and distinguished, especially when dealing with more than a single language --- and most crucially for theories comparing languages. Such meta-concepts might seem like an esoteric subject of high theoretical abstraction, and should thus probably better be left for late hours drowned in ample wine, weren't it for the fact that some widespread confusions in current linguistics are based on insufficiently clarified meta-concepts.

To make it easier to talk about these abstract tertiary concepts, I will propose concrete terminology for the various linguistic meta-concepts, like for example `languoid' or `counterpart'. The terms that I will use are sometimes newly coined (like `doculect'), but will mostly be drawn from the existing English lexicon (like `construction'). The danger of using existing words is that they will always evoke different denotations and connotations with different readers. Yet, whatever somebody might consider to be the meaning of a word like `construction', I invite any reader to make an effort not to mingle their own understanding of such a term with the proposals made here. In effect, an explicitly introduced technical term like `construction' should be read as a mnemonic device for a concept that might also have been called `gobbledygook' or which could be referred to as `Concept~\ref{construction}' (because all explicitly defined meta-concepts will be numbered throughout this book). Whether the terms themselves are well-chosen might be criticised, but any such discussion is actually `just words'. It is the underlying concepts and conceptual distinctions on which I invite scholarly scrutiny and discussion.

Because of the meta-meta-meta-level of each sentence in this book, the reading might at times be terse and demanding. But also the writing itself was tedious. Every word that I tried to use seemed to open up yet another can of worms of scholarly debate, or backfired with resonance onto a different level of meta-ness. I have rephrased almost everything multiple times to remove inconsistencies and to divert around pitfalls as much as possible. If the result feels artificially concocted, it is because I tried to be as concise and precise as possible. If the text does not run smoothly, a soothing prospect is that --- at least --- it is short.



\ \newline This book is a very personal take on linguistics, its conceptualisation and its terminology. If this view offers an insightful outlook, it is only because I am standing on the proverbial shoulders of giants while writing this. Extensive discussion with Martin Haspelmath, William Croft, Leon Stassen, Balthasar Bickel, Bernhard Wälchli, Johann-Mattis List, Steve Moran and Jeff Good have had an enormous impact on my thinking, and more often than not they might recognise their ideas in this book. I have tried to give the credit where I remember it, and if I forgot to mention any, I know you will take it as the sincerest form of flattery.

The main goal of this book is not to retell the fine minutiae of the scholarly discussion about the terms and the concepts presented. Instead, I will to try to present a more or less coherent and interconnected ensemble of linguistic meta-concepts. I will thus not go back to Aristoteles and provide a detailed discussion about who all else used comparable terms (though often with different conceptualisations) nor who all used similar concepts (though often using different terms), nor who had exactly which influence on my thinking as presented in this book. The main text will be just me talking, though I will add pointers to personal communication and to previous discussion in the literature by using footnotes.