\Chapter{Shape, strategy and pattern}{The structure of a typology}
\label{ch:typology}

The central problem of \textsc{linguistics}\refc{language} is to compare constructions from different languages to each other. This comparison poses a problem in a very trivial sense, namely in that linguistic elements from one language are different objects from elements in another language. A conceptually simple comparison across languages (though empirically quite tricky to establish) is to investigate whether two constructions are historically related. For example, the English suffix \textit{-ly} is historically `the same' element as the German suffix \textit{-lich}. These constructions are \textsc{homologs}\refc{homology}. However, such a comparison is only possible for a rather limited number of constructions, because either the languages in question have to be genealogically related, or the construction has been borrowed from one language to another. These kinds of comparison will be discussed in more detail in Chapter~\ref{ch:history}. This chapter deals with a more abstract kind of comparison that has become known under the term `linguistic typology'. Starting from a universally applicable \textsc{domain}\refc{domain}, this approach to language comparison defines universally applicable \textsc{typological characteristics}\refc{characteristic} to compare constructions across languages, leading to a \textsc{typology}.

\concept{typology}{
A \textsc{\textbf{typology}} is a \textsc{metric}\refc{metric} on \textsc{counterparts}\refc{counterpart}. Such a metric is based on \textsc{typological characteristics}\refc{characteristic}. In the most basic instantiation, a typology is just a partitioning of all counterparts into a few discrete groups, so-called \textsc{types}.
}
%
Counterparts are the linguistic expressions from different languages to be compared to each other (for more details, see chapter~\ref{ch:languagecomparison}). The actual comparison of these counterparts can take many different manifestations. In the most abstract sense, any such comparison is a kind of metric, i.e. the mathematical concept of distance (or its inverse, similarity). Counterparts can be more or less similar to each other, and the decisions about their similarity constitutes the typology. In the simplest possible metric, all comparisons between  counterparts lead to only two outcomes: either `the same' (i.e. the counterparts belong to the same type) or `different' (i.e. they belong to different types). A more extensive discussion of different kinds of metrics can be found below\refc{metric}. 

\concept{characteristic}{
A \textsc{\textbf{typological characteristic}} is a universally applicable property of a counterpart. There are at least three different kinds of such properties: \textsc{shape}\refc{shape}, \textsc{pattern}\refc{pattern} and \textsc{strategy}\refc{strategy}.
}
%
To establish a typology in practice, all counterparts are analysed according to some  typological characteristics. Many typologies are based on just a single typological characteristic. So, a single typological characteristic that distinguishes three different strategies\refc{strategy} trivially leads to a typology with three types. However, many different typological characteristics about the same counterparts can also be combined into a \textsc{multivariate typology}.\footnote{Bickel} Looking forward, this seems to be a much more fruitful approach, namely to start off distinguishing many different, often very simple, typological characteristics, and to derive more complex typologies on that basis.

In its most extreme form, a typology might include very many different, highly detailed typological characteristics, which might lead to a situation in which all counterparts are different from all other counterparts. In such a typology each language is its own type, leading in effect to a \textsc{typology without types}. Such a typology might seem to defy the rationale of language comparison, because it only seems to tell us that all languages investigated are different. However, not all counterparts are equally different. In a highly detailed multivariate typology some counterparts are more similar to each other than they are to others. This metric on counterparts represents an empirical restriction of the theoretically possible linguistic diversity.

\concept{shape}{
A typological \textsc{\textbf{shape}} is a cross-linguistically applicable characteristic classifying the \textsc{form}\refc{form} of a counterpart.
}
%
There is only a very limited number of possibilities to compare the form of counterparts across all the world's language. There are at least four different kinds of typological characteristics of form (and possibly not more): (i) size of constructions, (ii) quantity of counterparts, (iii) relative position construction parts, and (iv) extent of morphological fusion. First, the \textsc{size of a construction} can, for example, be measured in number of phonemes or in number of coding devices\refc{codingdevice}. Low values on such a measurement indicate that the domain under investigation is deeply ingrained in the structure of those languages (`strongly grammaticalised'). In reverse, a large size of a construction indicates that some form of circumlocution is apparently necessary. Second, the \textsc{quantity of counterparts} is a measure for the explicitness with which a language deals with the domain investigated (`paradigmatic diversity'). Some random examples are the number of phonemic vowels, the number of different cases, or the number of different past tense constructions. Third, the \textsc{relative position of construction parts} is of course best exemplified with the infamous word-order typologies. For such a characteristic, the comparative concept\refc{comparativeconcept} needs to specify different parts in each counterpart whose order can be investigated, like a possessor and a possessee in the domain of possession. Finally, the \textsc{extent of morphological fusion} is a characteristic typically used to distinguish morphologically bound marking from marking by separate words (`synthetic vs. analytic marking'). For example, in the domain of tense marking, a common distinction made is between marking tense as a morphological category on the verb vs. marking tense using separate particles or adverbs.

\concept{pattern}{
A typological \textsc{\textbf{pattern}} is a metric \emph{within} a counterpart. To establish a pattern it is necessary that the domain is subdivided into \textsc{subdomains}\refc{subdomain}. The similarity between the language-specific expressions of these subdomains represents the pattern of the counterpart.
}
%
Patterns are very powerful typological characteristics, which also play a central role in semantic maps (see Chapter~\ref{ch:semanticmaps}). The basic idea can be exemplified by the hand-arm polysemy as attested in various languages.\footnote{Ref to WALS} This characteristic can be analysed as being based on two subdomains, roughly identified as the body~parts called \textit{hand} and \textit{arm} in English. The expression of these two subdomains can show two patterns: either the two subdomains are expressed identically, or they are expressed differently. A second example is notion of ergativity, based on three subdomains, leading to five different possible patterns. Ergativity is based on the subdomains commonly called S(ubject), A(gent) and P(atient). Depending on the polysemy between the  constructions used for these three subdomains, there are theoretically five different possible polysemy patterns: nominative/accusative (A=S≠P), ergative/absolutive (A≠S=P), tripartite (A≠S≠P), neutral (A=S=P), and the apparently unattested transitive/intransitive role marking (A=P≠S).\footnote{Sapir 1917}

The number of patterns grows very fast with the number of subdomains. Two subdomains allow for two patterns, three subdomains for five, four subdomains lead to 15 different patterns, five subdomains generate 52 different patterns, and with 10 subdomains there are already more than 100.000 different theoretically possible patterns. The number of patterns for $n$ subdomains is the so-called $n$th Bell Number $B_{n}$. The size of $B_{n}$ grows very fast, even faster than exponential.\footnote{There is no closed-form formula for the calculation of Bell Numbers. For all details, see the On-Line Encyclopedia of Integer Sequences at http://oeis.org/A000110.}

Another way to look at patterns is to interpret a single pattern as a highly detailed multivariate typology. A pattern can described as a large collection of characteristics, namely one characteristic for each pair of subdomains. For example, with 3 subdomains, there are $\frac{3*2}{2} = 3$ different \emph{pairs} of subdomains, or more general: with $n$ subdomains, there are $\frac{1}{2}n(n-1)$ different pairs. Specifying the similarity for each of these pairs completely describes the pattern. 

In the examples above, patterns were exemplified as polysemy-patterns between subdomains, i.e. each pair of subdomains either use the same construction, or different constructions. This is of course only the most basic approach: it is equally possible to use fractional values of similarity between two constructions (i.e. constructions within a single language can be more or less similar to each other). In the words of the definition of a pattern as formulated above\refc{pattern}, a pattern is a metric between the expression of the subdomains. Or, slightly reformulated, a pattern is a metric within a domain.

The empirical appeal of typological patterns lies in the fact that to establish a pattern it is only necessary to compare constructions \emph{within} a single language to each other. Patterns thus provide a ingenious solution to the problem of comparability, because no actual comparison between constructions from different languages is necessary.

\concept{strategy}{
A typological \textsc{\textbf{strategy}} classifies counterparts by other uses of constructions outside of the domain.
}
%
Strategies could also be called `centric patterns'\refc{pattern}, because with strategies only polysemy relative to a single central domain is investigated. 

The practice of developing a strategy
The basic approach is the following: the counterparts in all languages are investigated for other uses within the same language.

centric pattern. Basic idea: take counterparts, and look for other possible meanings expressed by these counterparts. Then classify these meanings (but do not explicitly broaden the domain)

\concept{metric}{
Metric: definition of distance (similarity) between elements. Specification traditionally as a formula, but in practice often a large table with all distances, often semi-manually specified.
}



distance matrix / similarity matrix
(Gramian, kernel)

